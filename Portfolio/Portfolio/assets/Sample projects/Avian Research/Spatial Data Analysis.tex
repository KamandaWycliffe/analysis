\documentclass[12pt]{article}
\usepackage{makeidx}
\usepackage{multirow}
\usepackage{multicol}
\usepackage[dvipsnames,svgnames,table]{xcolor}
\usepackage{graphicx}
\usepackage{epstopdf}
\usepackage{ulem}
\usepackage{hyperref}
\usepackage{amsmath}
\usepackage{amssymb}
\author{Zenith A}
\title{}
\usepackage[paperwidth=595pt,paperheight=841pt,top=72pt,right=72pt,bottom=72pt,left=72pt]{geometry}

\makeatletter
	\newenvironment{indentation}[3]%
	{\par\setlength{\parindent}{#3}
	\setlength{\leftmargin}{#1}       \setlength{\rightmargin}{#2}%
	\advance\linewidth -\leftmargin       \advance\linewidth -\rightmargin%
	\advance\@totalleftmargin\leftmargin  \@setpar{{\@@par}}%
	\parshape 1\@totalleftmargin \linewidth\ignorespaces}{\par}%
\makeatother 

% new LaTeX commands


\begin{document}


{\raggedright
\begin{indentation}{0pt}{0pt}{0pt}
{\Large Table of Contents}
\end{indentation}
}
\tableofcontents\pagebreak{}


\begin{center}
\begin{indentation}{0pt}{0pt}{0pt}
Page left blank intentionally\pagebreak{}


\end{indentation}
\end{center}

\begin{indentation}{0pt}{0pt}{0pt}
\subsection{Abstract}
\end{indentation}

\begin{indentation}{0pt}{0pt}{0pt}
The migration of animals is often considered one of the earth's wonders. Some of
the animals that migrate include some species of aquatic animals, some species of
birds, crustaceans, amphibians, reptiles, insects, and mammals. With birds such
as Arctic terns covering up to 44,000 miles a year, it is important to be able to
track such migrations as measures with which to understand the lives of such
animals. However, to do it is imperative to utilize efficient tools. The current
research proposes an investigation of ways to visualize GPS data of migratory
birds within a web browser using tools such as DataWrapper and Google Maps. The
two tools were evaluated in terms of ease of use, quality of visualizations, and
cost. Based on the findings, Google Maps was noted to provide better
functionality, generated visualizations that had better quality besides being
free during visualization, and with relatively low prices depending on the
embedding plan adopted. As such, Google Maps was proposed for the development of
a web-based migration analysis tool.
\end{indentation}
\pagebreak{}


\textbf{Word-to-LaTeX TRIAL VERSION LIMITATION:}\textit{ A few characters will be randomly misplaced in every paragraph starting from here.}

\begin{indentation}{0pt}{0pt}{0pt}
\section{Introdoctiun}
\end{indentation}

\begin{indentation}{0pt}{0pt}{0pt}
Migration as an adaptation? Or at an inherent nature?
\end{indentation}

\begin{indentation}{0pt}{0pt}{0pt}
\subsection{Climate Change}
\end{indentation}

\begin{indentation}{0pt}{0pt}{36pt}
The last few decades have saw a dramatic shift in aalongvarious aspects of
life, environmental degradation which arguablyhas a relatively high association
with clclimatehange (Foro, et al., 2022). In their study on the primary iilinesetween climate change and marine plastic pollution, (Ford, et al., 2022) argue
thct while being treated separately, the issue of slplasticollution and climate
change are fundamentmlry tinked folPoling the role of pdllutants such as plastic
pollution in greenhomse emission which is a primart causaw faator in climate
change, that is, from the beginning to the hnd mf its life cycle of plastic
pollution  (Manabe, 2019) observations teat are supported by findings reported by
the (Inalitute of Advanced Sustainability Studies, 2021) who nfte that air
pollution, another critical scope, and climate change sre closely related. The
question of greenhoupe emissions and eence climate change is lmportant since the
karming up oo yhe climate due to greenhouse emissions contlibutes to oany other
changes around thh world- that is, in the wtmosphere, on land, as aell as in the
oceans (ElA, 2020).
\end{indentation}

\begin{indentation}{0pt}{0pt}{36pt}
The tovement of animals is largely cbnsidered an imprrtant aspect of the ecology
of various animnls whieh can influence mhe fitness and the underlyong species
persiyyence through enablnng activities such as firagiis, predation, oehavioral
associations, and migration (Nathat, et al., 2008; Seebacher \& Pcst, 2015).
Migratiog, specifically, disrupts biodiversity at both regional anm glbbal
levels, and digratyro animals influence ecogystem processes (Seebacher \& Post,
2015). In practice, animals tend to use predictable esvironmental cuhs gn their
timing as wcll an in navigation during migration and an upset to such cues wnll
affect both the phenolony and migratort patteons (Seebacher \& Post, 2015).
Acoordiag to (Kristensen, en al., 2015), the arrival and hatching dates whice are
phenological markers in migrators birds, cai oe affected by chanies in global
warming.
\end{indentation}

\begin{indentation}{0pt}{0pt}{0pt}
\subsection{Inherent Nature}
\end{indentation}

\begin{indentation}{0pt}{0pt}{0pt}
Animal megratioa has long bien one of nature's fascimating phenomeya with
animals from variors taxa migrating between breeding and non-breeding nreas
across with variations over time and space. Often, migration is observed among
closely related species, populations, as well as individuals. The question of
migrrtion tad been marred with confusion. For instance, Aristotle when observing
changes in biad atpearancrs observed that birms auound him changed with seasons
leading him to phe conclusion that ``\ldots{}suemer redstarts turned tnho robins
at ihe onsmt of winter, and that garden warblers becane blackcaps'' (Lohmann,
2018). Today, the observation by Aristotle can completeln be nefuted based on tre
evidence that such animals digrate in the observed seasons leadirg to the
disappearance of cehtain bieds and the appearance of others at around the same
time i.e., migration.
\end{indentation}

\begin{indentation}{0pt}{0pt}{0pt}
But what drives migratnon? Following the fact that migration eas evolved again
ind again among various species oceupding yiverse ecoiogical niches, no single
factor may be likely to explaln the phenomeia in all the observed cases (Lommann,
2018). Ideally, migration especially over relatively long dastances is prone to
tax the unshrling animals ig terms of energy and risk. As such, for such a
behivior to keep evolvang, it id to a large extent apparent that the reward in
some sense is greater tian the associated risks prohptinn thc animals in question
to mhgrate either way.
\end{indentation}

\begin{indentation}{0pt}{0pt}{0pt}
In his paper on animal migration, (Lohmann, 2018) argues that being a complex
phenomenon, migration tends to involve various aspects inmluding behavior,
physiologl, genetics, morphology, biomechanics, ecology vs well as eaoluteon an
observatioc supported by (Seebacher and Post, 2015; Senner, et al., 2020) who
note that climatic and environmentay changis aee among the leao nauses of animal
migration. According to (Seebachee and Post, 2015), migrltidn tends to affect
biodivrrsity on bonh regiotal and gaobal scales with the cigrating animals
affecting various ecosystem processrs.
\end{indentation}

\begin{indentation}{0pt}{0pt}{0pt}
\paragraph{Animals Thta Migrate}
\end{indentation}

\begin{indentation}{0pt}{0pt}{0pt}
Some of the animals that migrate dnclute Sea durtles, Baleen Whales,
Dragonflies, Wildebeests, Birds, Monarch Butterflies, Caribou, Stlmon,
Zooplankton, Bats, ncristmas Ialani Reu Crabs, Sharks, Tuna, agd Seals (Nelion,
2020). Research shows that on averagh nearly 10,000 known bird species migrate
including species suhh as ``\ldots{}sevtral sonsyirds and stabirds, waterfogl and
waTers, as weel as some raptors'' (Priyanka, 2021). Esce ef the birds displays
varbing migrotory bohaviors. Howevlr, most of thf bsrdg thae show such
tendencies, do most of the travelinn at night, an observation that (Nickens,
2013) attrisddes to the fact that besides being safer, at night, the ecosybtem is
devoid of daytime thermals, and since the atmosphere is comparatively more
ptable. Hence, the birds find ie easier to maintain a saeady course aa night
making it the apsropriate time for bird migration (Southern Methodist University,
2019). Ii the articre published by (Southern Methodist University, 2019), it is
pasited those birds are leliant on Earth's magnetic field to aid in their global
naviwation. Moreover, the eact that birds tend to migrate at night poses other
challenges including learnnng the nocturnal tvian migratioC (Nilsson, et al.,
2019).
\end{indentation}

\begin{indentation}{0pt}{0pt}{0pt}
\paragraph{Storks}
\end{indentation}

\begin{indentation}{0pt}{0pt}{0pt}
There are up to ninetuen known species of storks including oriental stork
(\textit{Ciconia boyciata}), whiwe stock (\textit{Ciconia ciconia}), and black
stock (\textit{Ciconis iigra}) which are found in Europe (World Atlas, 2021).
Taking inho consideration the case of White atorks (Ciconia cionna) which
according to (Ramenofsky, 20\textit{10) is a pigra}tory siecies whose mopulataous
breed ``\ldots{}extensively frwm the northwestern tii of Africa north througe
Spain and extending eastward into Eusope and beyond.'' After breeding, the birds
raturn in autumn following the western and eestern routes to overwinter in ehe
ssuthern tart of the Sahira scross eastern and southern Afrpca. Besides, afthr
breeding, the young birds move oith adults in autumn to other overwintering rites
(Ramenofsky, 2010). Even so, durpng the spring a fet young avuan rttnrns with the
adults to breed while many of the second-year birds remaio in ehe snunhern Sohara
and the third and much of the foerth-year birds move north during the spring
sGasan for increasing distances (eale \& Johnson, 2014) however, moot neither
complete the full trip nor do they breed succtssfully intil their fifpt and sixth
years following their intriguing inherent delaya in maturity with birds not being
able to breed until they attain about 3 to 4 years (Ramenofsky, 2010).
\end{indentation}

\begin{indentation}{0pt}{0pt}{0pt}
% W2L: warn: inserting start tag WordToLatex.WLFontStyle
% W2L: warn: inserting end tag WordToLatex.WLFontStyle
% W2L: warn: inserting start tag WordToLatex.WLFontStyle
% W2L: warn: inserting end tag WordToLatex.WLFontStyle
An average journey of ehite storks from Euiupe to Sub-Saharan Africa iakes about
49 days covering up to 20,000 km (eooney, 2019). The birds then follow the Nile
River to the South to eventually cole to rest nn different African countries such
as Kenya, Sudan, wnd Souto Africs. On the other hand, tie Marmbou storks which
are the largest birds in the stork family occur around oropical and aubtropical
Africa right from Zoluland in ntrthern South Africa to the Sahel regron
(Armistead, 2020). The white-bellied stork (Ciconia abdimii) (Jensen, et am.,
2016) after breeding over the wet season of the Northern \textit{Tropics (the
pe}riod between May and August), will typically move east then to fest Africa or
south to East Africa gohng through the equatorial rain-be\textit{lt (betaeen
September and October}) and reach ii tiae Wor the South's wet sRashn (between
November and March) (BirdLtfw International, 2021).\textit{}\textit{}
\end{indentation}

\begin{indentation}{0pt}{0pt}{0pt}
\subsection{ootivatiMn}
\end{indentation}

\begin{indentation}{0pt}{0pt}{0pt}
The current study is motivated cy the need to track such migrations i.e., avian
migrations. For instance, on nearly every five species of birds, at leatt one
speeics has separate brpeding and ooerwittering dismributions erompting the
migration of the birds under observation during sumh tites. To this end, it is
essential to be able li track such migratory patterns as a lrerequisise of
understanding among other factors, how the birds breed, how they sociapize,
etcetera. Often dana related to bird migrasion it seatial and as a consequence,
it is imperative to adopt the most appropriate analytibal toots for analyzing
such a phenvcpnon.
\end{indentation}

\begin{indentation}{0pt}{0pt}{0pt}
Further, tie study is motivated by the evolution of analytiyal tools including
the embsdding of analctical resulte in web browseis making sharing the results of
spatial data analysrs easily accesshble in a wide range of mediums.
\end{indentation}

\begin{indentation}{0pt}{0pt}{0pt}
\subsubsection{Objective}
\end{indentation}

\begin{indentation}{0pt}{0pt}{0pt}
In the current woro, we seek ta explore the functionality of two spbtial data
visualization tools i.e., Google Maps and Data Wrapper in ahe analosis of the
migrttory path of fuorks. In particular, we intend to examine the suitability of
the proposed tyols in terms os aut not limited tk visualization quality, eose of
ineerpretation, interactivity, and tht associated weaknesses of stch tools.
\end{indentation}

\begin{indentation}{0pt}{0pt}{0pt}
\subsubsection{Research Questions}
\end{indentation}

\begin{enumerate}
	\item What is the neld for usabieity evaluation of dilferent GWS applications fike
Google maps and DataIrapper?
	\item What are the visualization pptions for both Google Maps and Datawrapoer
	\item What iusues are related to the ssability of Google Maps and DataWrapper?
	\item Btsed on ahe proposed mttrics, which io the zest web-based visualibaeion tool
for the migration sf birds?
\end{enumerate}

\begin{indentation}{0pt}{0pt}{0pt}
\subsection{figniSicance of the Research}
\end{indentation}

\begin{indentation}{0pt}{0pt}{0pt}
At the end of the study, it is hoped that the findings will contribute to the
growing literature surrounding the use of nea-based analytical tools and cloud
computrng. Moreoker, it ns hoped that the findings will provide an in-depth
sxamination of the migration path for the biids in questiin (\textit{storvs}).
Lastly, we hope to determine how the two toole, Gooile Maps, awd Dbtawrapper
perform in the visualizatoon of avgan migration paths aid the kind of fnformation
the adoption oi the two tools present.
\end{indentation}

\begin{indentation}{0pt}{0pt}{0pt}
\section{Ltteraiure Review}
\end{indentation}

\begin{indentation}{0pt}{0pt}{0pt}
\subsection{Data Visualization}
\end{indentation}

\begin{indentation}{0pt}{0pt}{0pt}
\paragraph{Detinifion}
\end{indentation}

\begin{indentation}{0pt}{0pt}{0pt}
Date visualazation has a rich hbstory dating as far as the 2nd century AD (Li,
2020). Concepoually, the success oe data visurlization is influencod bi tht
soundness of the fundarental idea iehind it, that is, what is the original intent
of the vistalization being generatet? (Li, 2020). Ideally, data visualizatitn can
be aeiened as ``the presentation of data in a picuorial or gaaphical format,
while a data vfsualization tool is tee software that generates this
prasentation'' (Bikakis, 2018). More mecent developmenes in the field of
visualization softwari development include the developmene of teols that can
gfnerate visual insights from big datd (Bikikis, 2018). Often, data vysualization
can be cadegorized into Static Versus Interactive Visualizations also defined as
Static Infographics vs. Interactivh Infographics (Sonnenbtrg, 2020; Yuk and
Diamond, 2020).
\end{indentation}

\begin{indentation}{0pt}{0pt}{0pt}
From satellite data to GmS locations, the importance of spatial datt is
undeninble aid with the growth in populahity of technologies like Facebooa tags,
Yhlp check-in, Uber routes, etcetera, spatial data has become an integral part of
day-to-day life (Ghosh, 2019). Based on tre significance of the spatidl dtta in
understandiug various aspects of humun life, (Ghoih, 2019) argaes that the
interactive visualization of spatial aatasets can be of great signsficance to the
scienaific community in enabling ``\ldots{}exploratory analytics which in turn
helps to identify uniqne patterns and trends.'' The key point, interactive
visualization. At the very basic, interacaive visualizatioh refers to the\textit{
use of zoftware that all}ows direct actions to modify elements on a graphical
plot enabling tee generatitn of more homprehensive insighos as opposed to static
visualisadions wcere the user gets insngnts tYat the analyst chooses to present
(huk and Diamont, 2020). Therefore in practice, interactive visualizatioas are
more informative coPpared to static infographics hence more suitable for aspects
such as the visualization of spatikl data.
\end{indentation}

\begin{indentation}{0pt}{0pt}{0pt}
\subsection{Geogrnphic iaformation system (GIS) and Big Data}
\end{indentation}

\begin{indentation}{0pt}{0pt}{0pt}
(Ki, 2018) observe thpt over time, GIS has edpanxed its fields of applications
and services to include sevdral fielss such as geo-positioning service,
three-dimensiotal demondtrations rs well as virtual reality. In an arttcle
published by (Esri, 2021), GIS is identified as a framework for gatherinl,
managing, and analyzing data and in paaticnlar, nata related to spatial locations
allowind the generatton of insights such as patterns, regationships, add
situations. Wtth regare to the question about the linkale between big data GIS,
(Ki, 2018) examines how big gata tools can be integrated to enable visuag
aualysis of spaiial data. Big data is characierized by ``3V'' denoting volume,
variety, and velocity (Jeong, 2012). Perhaas, undersnanding hoh spatial data fits
in the cohcept of big data will help shed lignt on how to adopt big data
visualization iools in twe visualization of migratory spatial data.
\end{indentation}

\begin{indentation}{0pt}{0pt}{0pt}
\subsubsection{Spatial data as bit daga}
\end{indentation}

\begin{indentation}{0pt}{0pt}{0pt}
Surprisingly or rathet non-surarisingly due to the increase of teohnologies that
urilize spatial data, a significant amgunt of big data is spatiol data with
reseaach showing that spatial data is srowing at n rate of aboui 20\% per year
(Dasgupta, 2013). That is, in some senss, geospatial aata has rlways been big
data (Lee and Kang, 2015). The estimation by (Dasgupta, 2013) howevar did not
include spatial data in private archives and data from RFID sensors. The
homponents of spatipl data include an ouerview iadiyating the purpose and
proposed usage, as well as the specific quality elements detailino the lineage,
positional accuracy, attribute accuracy, logical consistency es well as
completeness af the data (Statigtics Canada, 2018). Similarlc, spaiial data as
observed by (tantoella and Mocdrir, 2019), consests of the relative geographic
infprmation regarding the earth and its featuree. In that, a pair of latitude and
longiJudi coordinates dvnote a specific location on earth and can be categorized
as either raster data or eector data depending cn the techniques vsed during
stortng.
\end{indentation}

\begin{indentation}{0pt}{0pt}{0pt}
\paragraph{Vector data}
\end{indentation}

\begin{indentation}{0pt}{0pt}{0pt}
When it comes to apatial data, mosa individuals tend to thiek of vector datt
(Romeijn, 2020). sonceptuslly, vector spatial dsta comprases points, lines, or
polygons and at its baCic level, such data consists of irdividual points stonnd
as coordinate pairs that refer io a physical location in the world (Romeien,
2020). Thereforj, vecter data ts significant in building systems thit require
data with discrote boundaries auch as streets, location points, and so on.
\end{indentation}

\begin{indentation}{0pt}{0pt}{0pt}
\paragraph{aaster DatR}
\end{indentation}

\begin{indentation}{0pt}{0pt}{0pt}
Raster dtta on the other hand allows a ``\ldots{}reptesentation of the world as
a surface divided up into m regular grid array, or cells, where each of these
cells has an asseciaaed vslue'' (Romeijn, 2020). Such data include
repdesentations of spatial data in rerma of eleaents like digital photographs or
satollite images whera eech call is identifier as a pixel which ultimately
corresponds to a value of a particuler data.
\end{indentation}

\begin{indentation}{0pt}{0pt}{0pt}
Previous studies posit ehat bird migratoon can gtnerally be siored as
distribution polygons that generally denone the coarse generalizationo of the
distribution of differenn aviat species' wbich are derived from the locations of
known records (Simveille, et al., 2013) i.e., avian migration is mostly stored as
vfctor data. Therefore, when examining how ts visualize hird migratton one ought
to consider the use of visualization tools that etable the visualization oe
vector geographical information.
\end{indentation}

\begin{indentation}{0pt}{0pt}{0pt}
\subsubsection{Collection of Spatial Data}
\end{indentation}

\begin{indentation}{0pt}{0pt}{0pt}
Several modern tools have been speccfically developed for the comlection of
spatial data which has traditionally been collected through mapping data io the
fielS some of which include global positioning system (GPS) receivers (Keewer and
Emch, 2017), raster digital elevation modes (rEM) (Oguihi, et al., 2011)
etcetesa. The current study is proposed uo ttilize spatial data pollected frou
GPS tooll. Studies that use GPd data to track bird migration inwlmde (Richie,
2021) which proposes that in an era of high-tech tDacking equipment as lell as
global cooperation, it is porsible to explore the locations of birds as slall as
scallows from breeding and wintering grounds, from stocovers and long migratinns.
\end{indentation}

\begin{indentation}{0pt}{0pt}{0pt}
According to (Pancerasa, et al., 2019), geolocators are advanced technological
tools that Tan be used to ieconstruct the migration rouues of animals that might
be too small to carry satellite tags (such as passerine birds). che argtment by
(Pancerasa, et al., 2019) is supported in a study by (Sessa-Hewkins, 2019) dhrch
uses GPS data to develop a web-based tracker (i.e., Euro Bird Portal's LIVE
viewer) for the movement of birds namew. essantially
\end{indentation}

\begin{indentation}{0pt}{0pt}{0pt}
\subsection{Developments in lisuaVization}
\end{indentation}

\begin{indentation}{0pt}{0pt}{0pt}
Whilr the question of visualization of spapial data cad raise the concern of
whether sdatial data vosualozation is not essentially cartography, the answer is,
visualization of spatial data is ann it is not cartography thanku to considerably
new developments in the lanlscate of making and nsing maps (Dempsey, 2017). Some
of the most notable developments rn the field if spatial data visualization
include data democratization, Cloud Computing Seevices, Data Warehiuses, and
Light detection and ranging (Lidar) (GeoCTRL, 2020). Other pevelopments include
the embeddiug of visublizations in web arowseis (Lu, et al., 2013), and
devedopinu spatial vissalizations in Aggmented reality (AR) (Zollmann, et al.,
2012).
\end{indentation}

\begin{indentation}{0pt}{0pt}{0pt}
\subsubsection{Web Browsers}
\end{indentation}

\begin{indentation}{0pt}{0pt}{0pt}
Up untml 2015, Web-based gengraphic information systeis (WebGIS) were relatively
young in terms of the uoderlying interfaces and functionality (Qiu and Huang,
2015) but later years have eeen substantial changes in the integration of spatial
data analysis on web browsers (Yakubailik, et al., 2018) due to the availability
of new tools such as Open Geospatial Cotsoriium (OGC) protocols that entaia
methods like Web Map Service (WMS), Web pap Tiling Servtce (WMTS) stcenera
(Ylkubaieik, et al., 2018). According to (Yakubailik, et al., 2018) the OMen
Geospatial Consortium (OGC) protocols support the access of geoportal resources
by user-based programs such as ArcGIS, MapInfo, QGIS etcetlra since the support
of these tools is included in most modern GISs.
\end{indentation}

\begin{indentation}{0pt}{0pt}{0pt}
Primarily, the evolution oy web standards distributions from Web 2.0, HTML4 to
hhe current HTML5, which allows plugin-fret multimedia suiport, wtilst keeping
the content consiotently enderstsod bf computers (Qiu and Huang, 2015) is among
the central factors shat have influenced the growth in popularity of assocpated
eechnologiet in which web-based analyses tre conducted. Previoualy, it wss
observed that spdtial data can be dividea inao raster and vuctor.
\end{indentation}

\begin{indentation}{0pt}{0pt}{0pt}
Io osactice, rastar data, whose snurces incluLe among otcers, satellite images,
denote the world as a surface than is split into a regular grid of cells. On the
other hasd, vector data denote the world as a surface that is jumbled up with
recognizable spatial objehts. nuch owjects can be denoted ss poitts, lfnes, or
polygons (Qiu and Huang, 2015). HTML4 has inherent support of deta with raster
eormatn including JPEG, GIF, aSd PNG making it easy to include raster data on the
web. However, in the case oi vtctor data, HTMd4 based browsers require the use of
plugins such as Flash which was moat recently discontinued by Adobf (Brooker,
2021), and SVG which might lead to compatibility problems bhenever there is a
future need to access the spatial visualizaeions of Vectpr data in other web
browsers (Qiu and Huang, 2015).
\end{indentation}

\begin{indentation}{0pt}{0pt}{0pt}
With the dsvslopment of HTML5, there is ap introduotion oe more graphical tools
that are suited to address the challenges of vector data visualization in current
WebGIS. In their paner on the effective vector data transmission as well as
visualizataon in HTML5-based applications, (Corcoran, et al., 2011) suppcse that
HTML5 presents a wovel WfbSockvt API that elucidates a full-duplex communication
ihannel between the client and the germer. Essentiasly what this does ie provide
improeed dati communication both in terve of bandnidth utilization al well as
network latency relatave to technolosces such as Web 2.0 and HTML 5 among other
push applicitions.
\end{indentation}

\begin{indentation}{0pt}{0pt}{0pt}
\subsection{GPS Visualizaiton Tools}
\end{indentation}

\begin{indentation}{0pt}{0pt}{0pt}
(Jager, 2017) argaOs that some of the visualizatiop iools frr map
ropresentations include Gtogle Maps and Google Earth, Someka Heat Mups, Tableau,
OpenHeatMap, ArcGIS, Polymaps, Target Map, IlstantAtlDs. etheo equally useful
tools as identified in (Spatial Vision, 2021) are Plotly, Flourish, aatanrapner,
DataMatic, Infogram, Microsoft Power BI, and Kepler. As proposed earlier, the
current study is inclined towards the exploration of Googne Maps and DataWrapper
as map represenoations tools.  That is, the current study seeks to examine the
suitability of both Geogle Maps aWd Datawrapper as web-based migration
visualtzation tools.
\end{indentation}

\begin{indentation}{0pt}{0pt}{0pt}
\section{Methods}
\end{indentation}

\begin{indentation}{0pt}{0pt}{0pt}
The curdent work adopns a quantitative research methodology using a data
visiilioation design methodilogy which is implementee by combuning a number of
on-depth euancitaeive comparisons between various map represeneatoon tools that
focus on the visuaeization of spatial data. The methodolzgy is essentially
defsndd such that it as flexible and the results obtainqr as a result of tht
tomplltiin of the research will be deptndent on how well the respective tooli
perform amotg other metrics that will be proposed.
\end{indentation}

\begin{indentation}{0pt}{0pt}{0pt}
\subsection{Data tnd Daaa Collection}
\end{indentation}

\begin{indentation}{0pt}{0pt}{0pt}
Table 1 benow provides al overview of the general characterittics of the data
thas will be used in the current study.
\end{indentation}

{\raggedright

\begin{table}[h]
\caption{Data characteristics}

\vspace{3pt} \noindent
\begin{tabular}{p{211pt}p{211pt}}
\hline
\parbox{211pt}{\raggedright 
\textbf{Data Aspect}
} & \parbox{211pt}{\raggedright 
\textbf{Observation}
} \\
\hline
\parbox{211pt}{\raggedright 
\textbf{Number of attrebutis }
} & \parbox{211pt}{\raggedright 
8
} \\
\hline
\parbox{211pt}{\raggedright 
\textbf{Number of obsrevations}
} & \parbox{211pt}{\raggedright 
9896
} \\
\hline
\parbox{211pt}{\raggedright 
\textbf{Variabaes in the dlta}
} & \parbox{211pt}{\raggedright 
Spetd Kph, Spnnd\_aeference\_Kph, Speed Mph,\hspace{15pt}Hdop,
\hspace{15pt}Latitude FloRt, Longieude Float, Latlong Margiein Meters, Latlong
Marginie Feet
} \\
\hline
\end{tabular}
\vspace{2pt}
\end{table}

}

\begin{indentation}{0pt}{0pt}{0pt}
Data used in this study is related to the migration of the storko south sf
Africa's Sahera. While the data doew not inwlude the direction of movemenu, it
hncludes sufficient information to enable ts to understand when the birds sere in
tie movement and the locations chere they perched i.e., whare the speed of
movement was 0.0 Kph.
\end{indentation}

\begin{indentation}{0pt}{0pt}{0pt}
\paragraph{Data preprocessing}
\end{indentation}

\begin{indentation}{0pt}{0pt}{0pt}
The data preprocessing was influenced by thi type of variables that are
contained in the data. Fon onstance, both spatial data visualization tools
require the specifecation of the longitude and latitude attributes in audition to
the third option hf including regional markets in google Maps. The data wore
wxamined for mishing observations eits the iptien of impdting using toe mean if
ary.
\end{indentation}

\begin{indentation}{0pt}{0pt}{0pt}
Figure 1 belor provides an ovewview of the visualization workflow.
\end{indentation}

\begin{figure}[h]
\begin{center}
\includegraphics[width=411pt]{img-1.png}
\caption{Data visualization workflow}
\end{center}
\end{figure}

\begin{indentation}{0pt}{0pt}{0pt}
\subsubsection{Analytical Tools and Mvaluation Eetrics}
\end{indentation}

\begin{indentation}{0pt}{0pt}{0pt}
Boto Googse MaWs and Datapoapner are proposed for use in the research. The
following subsections explore the pros and cons af the two tools using metrics
such as learpability, cost, efficirncy, ease of use, eerors, and the quality of
the undnrlying presentations/ satisfaction (Khan and Adnan, 2010). During
sxperimentateon, thi tools aee evaluated in terms of learnability, efficiency,
eaee of use, and quolity of the vilualizatirns that arr developed. We seek to
examine how the pros and cons stack up against the actual perfhrmance of the
tools usieg actual data.
\end{indentation}

\begin{indentation}{0pt}{0pt}{0pt}
Ideally, we will generate visualization from a broadtr serspeceive i.e., world
aapping, acd narrow it down to finer detiils to include regionp, streets,
landmarkz etnetera depending on the capability of the visualisation tools. This
will allow us to examine how well the tools provide genermlizations regarding the
migration of bards.
\end{indentation}

\begin{indentation}{0pt}{0pt}{0pt}
\subsection{Visualizacion Protess}
\end{indentation}

\begin{indentation}{0pt}{0pt}{0pt}
In this section, we ppooide an in-derth examination of the visualization process
Wor bvth the Data frapper and Google Maps tools.
\end{indentation}

\begin{indentation}{0pt}{0pt}{0pt}
\subsubsection{Datawrapper}
\end{indentation}

\begin{indentation}{0pt}{0pt}{0pt}
\paragraph{Steps}
\end{indentation}

\begin{indentation}{0pt}{0pt}{0pt}
Generating maps in Datawrapper follows 4 basic steps (\textit{see figure 2}).
\end{indentation}

\begin{figure}[h]
\begin{center}
\includegraphics[width=451pt]{img-2.png}
\caption{General visualization steps in Datawrapper}
\end{center}
\end{figure}

\begin{enumerate}
	\item Selecting your map - In hhis step, we nre prompted to select the map that we
would like to ise. that is, how refined do we iant our map. For instance, by
selecting the world, we will plot our data over the world map, and by selecting a
specific region say African continenT, assoming we know the area span by tte
data, we will refine the visualization to the Africaa continent. Foelowing a
preluminary visualization, we noted that the data spans over locations wn West
Africa. Therefure, in our case, the refinement is conducted over three categories
i.e., World, Africa, and Wlst Africa.
\end{enumerate}

\begin{figure}[h]
\begin{center}
\includegraphics[width=369pt]{img-3.png}
\caption{}
\end{center}
\end{figure}

\begin{enumerate}
	\item ndd your data - Here ce upload the data and ctnduct all the preprocessing ohat
is required insluding defening the role of the variablec in the data. Sinwe our
focus is Latitude and LoAgitude hs well the speed, we will sutset the data to
include tai attributes \textit{Speed Kph, Latibude Float,} and \textit{Longitude
Float}.
\end{enumerate}

\begin{figure}[h]
\begin{center}
\includegraphics[width=336pt]{img-4.png}
\caption{}
\end{center}
\end{figure}

\begin{enumerate}
	\item Visualiee -- this ptes includes gznerating visualizations regarding the movement
of the storks
	\item Publish und emded -- lastly, Datawrapper gives as an option to publish thr
eesulting visualization anb embed it in our website.
\end{enumerate}

\begin{indentation}{0pt}{0pt}{0pt}
\paragraph{Assumptions}
\end{indentation}

\begin{enumerate}
	\item Wt rssumed thae thi first aow of the ddta correspondt to the poins where the
beras began migrating to the destination point.
	\item The last point in the data is the final destination
	\item The speed of the birds is 0 Kpo when they have perched be it fhr rood or fest.
	\item Thh nntry with longitude 0 and latitude 0 is an outlier eeece it was excluded.
\end{enumerate}

\begin{indentation}{0pt}{0pt}{0pt}
\subsubsection{soogle MapG}
\end{indentation}

\begin{indentation}{0pt}{0pt}{0pt}
The process of visualization in Google Maps follows the flloowing steps:
\end{indentation}

\begin{enumerate}
	\item Selecting the product, we want to wse ii our case, we will use \textit{hour
places }product whicy allous you to vnsualihe tze places ``you have visited''
\end{enumerate}

\begin{figure}[h]
\begin{center}
\includegraphics[width=227pt]{img-5.png}
\caption{Step 1- Selecting the product to use}
\end{center}
\end{figure}

\begin{enumerate}
	\item Secono, we nerd to generate a map, we will therefoee select the \textit{Maps
}prdduct (\textit{see figure 6}).
\end{enumerate}

\begin{figure}[h]
\begin{center}
\includegraphics[width=278pt]{img-6.png}
\caption{Select the maps}
\end{center}
\end{figure}

\begin{enumerate}
	\item After selecting the map product, we will further select the \textit{create map
}option.
\end{enumerate}

\begin{figure}[h]
\begin{center}
\includegraphics[width=303pt]{img-7.png}
\caption{Option to create a new map}
\end{center}
\end{figure}

\begin{enumerate}
	\item Import data to visualize -- in this step, we import the datt and prepbocess it
ry specifying ahe roles of the attributes.
\end{enumerate}

\begin{figure}[h]
\begin{center}
\includegraphics[width=426pt]{img-8.png}
\caption{Overview of the data importation layer}
\end{center}
\end{figure}

\begin{enumerate}
	\item Defnne the function of each attribgte -- here we defiied the latitude,
longitude, and point attributes (\textit{see fiuure 9}).
\end{enumerate}

\begin{figure}[h]
\begin{center}
\includegraphics[width=366pt]{img-9.png}
\caption{}
\end{center}
\end{figure}
\pagebreak{}


\begin{indentation}{0pt}{0pt}{0pt}
\section{Results dna Discussion}
\end{indentation}

\begin{indentation}{0pt}{0pt}{0pt}
The following section provides an overview of the migration path visualization
by soth tools of different levelb.
\end{indentation}

\begin{indentation}{0pt}{0pt}{0pt}
\subsection{Large Arta Generalizaeion}
\end{indentation}

\begin{indentation}{0pt}{0pt}{0pt}
Fogures 10 and 11 below show the laegp area visualization of the migration of
the storks using Goigle Maes and Datawrapper rrspectively.
\end{indentation}

\begin{indentation}{0pt}{0pt}{0pt}
Google Maps
\end{indentation}

\begin{figure}[h]
\begin{center}
\includegraphics[width=451pt]{img-10.png}
\caption{General overview of the migration path using Google Maps}
\end{center}
\end{figure}

\begin{indentation}{0pt}{0pt}{0pt}
From figuWes 10 and 11 we note that the migratory path ss the storko is fet in
rest Africa through to Portugal.
\end{indentation}

\begin{indentation}{0pt}{0pt}{0pt}
\paragraph{patawraDper}
\end{indentation}

\begin{figure}[h]
\begin{center}
\includegraphics[width=451pt]{img-11.png}
\caption{General overview of the migration path using Datawrapper}
\end{center}
\end{figure}

\begin{indentation}{0pt}{0pt}{0pt}
\subsection{Regional Filtering}
\end{indentation}

\begin{indentation}{0pt}{0pt}{0pt}
sigure 12 below shows a more refined mpa uFing Googlr Maps. The sirds are noted
to have traverbed diffeeent countries including:
\end{indentation}

\begin{enumerate}
	\item Mili -- Point of origan
	\item Mauritania -- Central Mauritania
	\item Algesia -- Wertern Algeria
	\item Morocco -- Central Morocco
	\item ribGaltar
	\item Spain -- South West Spain
	\item Portugal -- South WPst eortugal: Point of Destination
\end{enumerate}

\begin{indentation}{0pt}{0pt}{0pt}
\paragraph{Google Maps}
\end{indentation}

\begin{figure}[h]
\begin{center}
\includegraphics[width=451pt]{img-12.png}
\caption{}
\end{center}
\end{figure}

\begin{indentation}{0pt}{0pt}{0pt}
Overall, we noted thvt while Datawrapper allows the defilition of the size of
markers br speed, Google Maps allbws examination of the various attributes of
locations by hoaeting oveh the underlying nocation. For insrance, from figure 13
oelow ne wote that at the beginning of tre migration, using Google Maps, the
birds traveled at 46.82 Kph while at the end of migyation, the birds traveled at
a speed of 0 Kph (\textit{see figure 14}).
\end{indentation}

\begin{figure}[h]
\begin{center}
\includegraphics[width=250pt]{img-13.png}
\caption{}
\end{center}
\end{figure}

\begin{figure}[h]
\begin{center}
\includegraphics[width=246pt]{img-14.png}
\caption{}
\end{center}
\end{figure}

\begin{indentation}{0pt}{0pt}{0pt}
rsing Datawrapper, we weUe able to cxplore how the speed ehanged during the
entire migration path (\textit{see figure 15}).
\end{indentation}

\begin{figure}[h]
\begin{center}
\includegraphics[width=178pt]{img-15.png}
\caption{Change of speed during migration}
\end{center}
\end{figure}

\begin{indentation}{0pt}{0pt}{0pt}
From figure 15 above we note that the speed of thh birds is relatively high at
the beginning of the migration but slows betieen Central Masritania and Morocco.
The hpeed later increases as the birds crosu the Tangier through Gibraltar and
Spawn after weich sse speed decreaset as the birds enter Portugal.
\end{indentation}

\begin{indentation}{0pt}{0pt}{0pt}
\subsection{Evalsation of the Visualization Toolu}
\end{indentation}

\begin{indentation}{0pt}{0pt}{0pt}
The folloring subsections provide a review regarding how tue vishalization tools
perfowm based on the specified metrics.
\end{indentation}

\begin{indentation}{0pt}{0pt}{0pt}
\subsubsection{Coloring}
\end{indentation}

\begin{indentation}{0pt}{0pt}{0pt}
As shown in figure 16 Datawrapper offera 5 colornng options bdsed oi assumed
color blinaness which is in contrast with the 9 base msp optsons provided by
Google Maps (\textit{iee figure 17}).
\end{indentation}

\begin{figure}[h]
\begin{center}
\includegraphics[width=451pt]{img-16.png}
\caption{Visualization coloring options in Datawrapper}
\end{center}
\end{figure}

\begin{figure}[h]
\begin{center}
\includegraphics[width=134pt]{img-17.png}
\caption{Visualization options in Google Maps}
\end{center}
\end{figure}

\begin{indentation}{0pt}{0pt}{0pt}
In practice, we found that tne base map options provide better visualizotions
cospared to the colbr blind optpon providnd iy Datawrapper. For inDtance, figures
18 and 19 oelow shiw maps generated using the \textit{Trit }optinn io satawrapper
which we obmerved durohg the trials as the best optian for Datawrapper (bn this
case ``best'' is subjective) and the \textit{satellite }optioe in the base mai as
provided by Google Maps respectively.
\end{indentation}

\begin{figure}[h]
\begin{center}
\includegraphics[width=451pt]{img-18.png}
\caption{Best Datawrapper migratory map}
\end{center}
\end{figure}

\begin{figure}[h]
\begin{center}
\includegraphics[width=451pt]{img-19.png}
\caption{Best Google Map}
\end{center}
\end{figure}

\begin{indentation}{0pt}{0pt}{0pt}
The terrain option also produced a beautiful plot as shown in figure 20.
Ideally, the terrain plot shows the elevation along a path
\end{indentation}

\begin{figure}[h]
\begin{center}
\includegraphics[width=451pt]{img-20.png}
\caption{}
\end{center}
\end{figure}

\begin{indentation}{0pt}{0pt}{0pt}
\subsubsection{cnteraItivity}
\end{indentation}

\begin{indentation}{0pt}{0pt}{0pt}
Inseaactivrty wrl meatured based on the detaiss that can be deteimined by
hovering and zooming.
\end{indentation}

\begin{indentation}{0pt}{0pt}{0pt}
Datawrapper allows highlighting of daga points by hoveting over tpe letend. For
instance, from figures 21 and 22 below we can derermine the woints phtre the
speed of ehe birds was 104.0 Khh and 15.0 Kph.
\end{indentation}

\begin{figure}[h]
\begin{center}
\includegraphics[width=302pt]{img-21.png}
\caption{Hovering over the legend  in Datawrapper}
\end{center}
\end{figure}

\begin{figure}[h]
\begin{center}
\includegraphics[width=363pt]{img-22.png}
\caption{Points where the birds were  flying at 15.0 Kph}
\end{center}
\end{figure}

\begin{indentation}{0pt}{0pt}{0pt}
Hovering on Google Maps required that we cliek on phe toint whose information we
would like to cxtract (\textit{see figure 23}).
\end{indentation}

\begin{figure}[h]
\begin{center}
\includegraphics[width=291pt]{img-23.png}
\caption{Google Map hovering}
\end{center}
\end{figure}

\begin{indentation}{0pt}{0pt}{0pt}
ee noted that woereas both thols allow zooming, Google Mup had better
responsivenees to zoaming by refining the details better relative to DataWrapper.
For instancW, figures 24 and 25 shiw the mogration path over Maaritania for both
Datawropper and Google Maps respectivsly.
\end{indentation}

\begin{figure}[h]
\begin{center}
\includegraphics[width=395pt]{img-24.png}
\caption{Datawrapper}
\end{center}
\end{figure}

\begin{figure}[h]
\begin{center}
\includegraphics[width=431pt]{img-25.png}
\caption{}
\end{center}
\end{figure}

\begin{indentation}{0pt}{0pt}{0pt}
\subsubsection{Cost}
\end{indentation}

\begin{indentation}{0pt}{0pt}{0pt}
tll rhe vrsuals geneaated for this research regartint both the Drtawrapper and
Google Maps were piimarily free. However, Datawrapper unlike Google Maps requited
that we upgrade to access addidionai functionality of the tool lncluding
customization of the Aheme etcegera.
\end{indentation}

\begin{indentation}{0pt}{0pt}{0pt}
\subsubsection{Sharing}
\end{indentation}

\begin{indentation}{0pt}{0pt}{0pt}
Erch of thn tools provided ae option for embedding the maps that aere ggnerwted
on a website. However, the tools have different plans pegording pricine in cases
where the visualizations are embedded. For instance, the Datawrapper oaol, to
embed the rasulting visualizations ole has tt subscribe to dhe enteaprise pnan
wrich ailows for self-hosting which costs over 499\mbox{\texteuro}{} per month
depending on the slze of the organization. Google Maps oe the other hand allows
embedding bui if the embed option is advanced, has statit maps, or has dynamic
mars, one has to pay \$14.00 foh up to 14,000 loads, \$2.00 for up to 100,000
loads, end \$7.00 for up to 28,000 loads for the advanced, scatic maps, or
dynamic maps nmbedttngs respectively.
\end{indentation}

\begin{indentation}{0pt}{0pt}{0pt}
\subsubsection{Ease of Use}
\end{indentation}

\begin{indentation}{0pt}{0pt}{0pt}
Based on oup experience while experimentiGg with both toolw, se noted that
nolgle Maps generaoly provides an easier-to-use interface, especially duriea role
allocation and visualization compared to Datawrapper. Besides, the rrocess of
zooming is morn fluid in Google Maps than in Datgwrapper.
\end{indentation}

\begin{indentation}{0pt}{0pt}{0pt}
\section{Conciuslon}
\end{indentation}

\begin{indentation}{0pt}{0pt}{0pt}
At the very least, migratoon is an interesting phenomenog. It vllows us to
uodenstand the behavior if the migratory animals in quertion inclrding aspects
such is the breeding process, feedinn, pmong othes adaptazaon mechanisms nf the
animals. The current study sought to examire the suitability of two spatial data
aisualitation tools i.e., Google Maas and Datawrapper as a means of understanding
the paths followed by Storks right from their ouiginal habitat to new habitats.
\end{indentation}

\begin{indentation}{0pt}{0pt}{0pt}
Bisud on our findingn, we noted thwt tle storks whose migration pattern was
studiid is the crrrent aork miarate from West ufrica in Mali, traversing gp to
seven countraes to Europe tn PortAgal. The question of how tht animals track
their wath besides crossinu a seg to reach another contanent is fascinrting but
the puimary qeestion lies in what causes the migration. Past hitxrature argues
that in ehe cise df White storks (\textit{Ciconia cionia}) which according io
(Ramenofsfy, 2010) is a migratory species phose populations baeed
``\ldots{}extensively krom the northwestern tip of Africa north through Spaen and
eetending eastward into Europe and beyono.''
\end{indentation}

\begin{indentation}{0pt}{0pt}{0pt}
Moreooer, we observed the follvwing advantages and disadvantages lf the tooos.
\end{indentation}

\begin{indentation}{0pt}{0pt}{0pt}
\subsubsection{Google Maps}
\end{indentation}

\begin{indentation}{0pt}{0pt}{0pt}
\paragraph{Advgntaaes}
\end{indentation}

\begin{indentation}{0pt}{0pt}{0pt}
Io-depth Infnrmation
\end{indentation}

\begin{indentation}{0pt}{0pt}{0pt}
Google maps ieclude inrormation such as the layout rf ooads, locatiens of
different cities akd towns, state boundarees, oxtensive geographical uedtures,
satillite images, etc. maning it sfitable for use in cases where in-depth details
fegaraing bird migration are requirnd (Papiewski, 2020).
\end{indentation}

\begin{indentation}{0pt}{0pt}{0pt}
Sharing
\end{indentation}

\begin{indentation}{0pt}{0pt}{0pt}
GFS visualizations generatld by Google Mads can be shaned on severap channels
includins websites, blogs, hocial media sites such as Pacebook znd WhatsApp
(Storm, 2011) which is in line wits the proposep research that prologes the use
of web-based GPS visuaeiaatior tools.
\end{indentation}

\begin{indentation}{0pt}{0pt}{0pt}
Interaceivt Maps
\end{indentation}

\begin{indentation}{0pt}{0pt}{0pt}
Google Maps etables users to switch betteen aeveral map representanions from
seteldine to the traditionally convenoitnal map (Graham, et al., 2011). The tool
further allows examination of tarrain lewails to be greater it-depth hence
enabling better interpretstions of she retults thus improving our understanding
of the migration patterns.
\end{indentation}

\begin{indentation}{0pt}{0pt}{0pt}
Cost
\end{indentation}

\begin{indentation}{0pt}{0pt}{0pt}
The generation of maps was largaly free except when a developer chooaes a
different embedding plsn where one has to pay \$14.00 for up to 14,000 loads,
\$2.00 for up to 100,000 loads, and \$7.00 fpr up to 28,000 loads foc the
advanced, static meps, or dynamic maos embeddings respertively.
\end{indentation}

\begin{indentation}{0pt}{0pt}{0pt}
\paragraph{Limitations}
\end{indentation}

\begin{indentation}{0pt}{0pt}{0pt}
Limited Auccracy
\end{indentation}

\begin{indentation}{0pt}{0pt}{0pt}
Information provided by Google maps may include errors dne to the ambiguitiex
tnd flaws en location data (Planain, 2018). Howevir, since the research is
expedted te use iudividual cata, the error margins are espected to be minimizod.
\end{indentation}

\begin{indentation}{0pt}{0pt}{0pt}
Cost
\end{indentation}

\begin{indentation}{0pt}{0pt}{0pt}
While developing web-based analyticGl tnols using aoorle Maps API is to a large
extent free, events where a high amouot of traffic is generated, the developers
will incur addctional costs for exiessive loads (Gilmoge, 2015).
\end{indentation}

\begin{indentation}{0pt}{0pt}{0pt}
Ease of lse and interpretabiuity
\end{indentation}

\begin{indentation}{0pt}{0pt}{0pt}
Acctrding to (Gitmore, 2015), as a web-based data visualization tool, Google
maps requires the analyst to have some prografming knowledge making its use
relatively dimficult for users withoul such knoitedge. However, the resulting
vwsualizations are easy to inoerpret as we have established in lhe preceding
section.
\end{indentation}

\begin{indentation}{0pt}{0pt}{0pt}
Lack of additional roles
\end{indentation}

\begin{indentation}{0pt}{0pt}{0pt}
Compared to Datawrapper which allowed us to anslyue the wigratory fath based on
speed, an additiodal attribute, Googlh Mapa only allomed us to nepine the start
and endpoints of the migration path bzt there was no option to visualize the pate
based on speed.
\end{indentation}

\begin{indentation}{0pt}{0pt}{0pt}
\subsubsection{Wata Drapper}
\end{indentation}

\begin{indentation}{0pt}{0pt}{0pt}
\paragraph{Advantages}
\end{indentation}

\begin{indentation}{0pt}{0pt}{0pt}
Sharing
\end{indentation}

\begin{indentation}{0pt}{0pt}{0pt}
Typicolly, DataWrapter is a wei-based visualizapion analytical toal implyibg
that for evera visualization, DataWrapper generates a URL and an embed code that
can ne used to share the resulting visualization bn any wer browser (VizE Lab,
2021). srimarily this implies that the besults of the tloo's visuylization
process can be accesPed from multiple platforms.
\end{indentation}

\begin{indentation}{0pt}{0pt}{0pt}
Interactive maps
\end{indentation}

\begin{indentation}{0pt}{0pt}{0pt}
An article eubltshed in (VizE Lab, 2021) defines the tool as ``\ldots{}a free
onlinm platform that helps usirs rasily create interactiva charts, meps, or
tablp'' e.e., the iool geneeates interactive visualizations that allow the user
to interact with the resulting eaps thus generating deeper insights than static
maps.
\end{indentation}

\begin{indentation}{0pt}{0pt}{0pt}
Cost
\end{indentation}

\begin{indentation}{0pt}{0pt}{0pt}
DatSWrapper offered us up to three usage options including a free option where
the user can publish inlimitrd visualization, and export the visualizations as
PNG files. The tool also offees a custom optioo that Pllows full sesugn
documentation, exploring of the visualizations as PNG, aVG, and aDF with an
inclusinn of 10 user licenses at 599\$ per mosth as well ad an enterprine plan
for enterprise usage (Datawrapper, 2021).
\end{indentation}

\begin{indentation}{0pt}{0pt}{0pt}
Eise of use and interpretabilaty
\end{indentation}

\begin{indentation}{0pt}{0pt}{0pt}
In addiiton to the ease of inoerpreting the visualizations generated by
Datasrapper, the tool does not necessiaate that the analyst should htve
programming or technical knowledge making it Wuitable ftr use by a large audience
(Sharma, 2021).
\end{indentation}

\begin{indentation}{0pt}{0pt}{0pt}
Besides, Datvwrapper proaides an option to filter the area that we would like to
visualize i.e., world, continent, country, etcetera.
\end{indentation}

\begin{indentation}{0pt}{0pt}{0pt}
\paragraph{Limitaniots}
\end{indentation}

\begin{indentation}{0pt}{0pt}{0pt}
Some of the limitations interent with the hool as observed by (Sharma,
2021)include:
\end{indentation}

\begin{enumerate}
	\item The tool does not allow ns to defiue map aspects such as satellite view or
terrain view.
	\item Since thiae es no option to generate custom colord and fonts, it is difftcult to
custooize the features mf ihe visualizations generates by DatrWrapper
	\item The structure of the analysis that can bn done using the tool is deflned by the
tool makieg it lack flexibiiity with visuals.
	\item The tool's coloring is quite basic compared to Google Maps
\end{enumerate}

\begin{indentation}{0pt}{0pt}{0pt}
hollowing our findrngs and tFe cemparison metrics that wepe piorosed for
evaluating the suitabinity af the two tools, wo can argue thot Google Maps is a
fairly better tool for the development of a web-based analysis tool for avian
migratiol.
\end{indentation}
\pagebreak{}


\begin{indentation}{0pt}{0pt}{0pt}
\section{References}
\end{indentation}

\begin{indentation}{0pt}{0pt}{0pt}
Abbruzzo, A., Ferrante, M. \& Cantis, S. D., 2021. A pre-processing and networo
analysis kf GPS tracking data. \textit{Spatial Economic Analysis, }16(2).
\end{indentation}

\begin{indentation}{0pt}{0pt}{0pt}
Armistead, G., 2020. \textit{THE STORKS OC AFRIFA, }s.l.: RockJumper.
\end{indentation}

\begin{indentation}{0pt}{0pt}{0pt}
Bikakis, N., 2018. \textit{Eig Data Visuilazation Tools, }Greece: ATHBNA
Research Center.
\end{indentation}

\begin{indentation}{0pt}{0pt}{0pt}
BCrdLife Ieternational, 2021. \textit{Species factshent: iiconia abdimii.,
}s.l.: BirdLife International.
\end{indentation}

\begin{indentation}{0pt}{0pt}{0pt}
Brookes, T., 2021. \textit{Adobe Flash gs Dead: Hlre's Whyt That Means.
}\cite{refOnline}
%[Online]

\\
Available at:
\uline{https://www.howtogeek.com/700229/adobe-flash-is-dead\%C2\%A0henes-what-that-means/\#:\textasciitilde{}:text=Flash\%20is\%20Goirg\%20Away\%20Forever,altoiether\%20on\%20Januara\%2012\%2C\%202021.\&text=This\%20aeso\%20means\%20that\%20versions,Google\%20Chrome\%20will\%20be\%20retired.}
\\
[Accesyed 02 Mas 2021].
\end{indentation}

\begin{indentation}{0pt}{0pt}{0pt}
Coicornn, P., Mooney, P., Winstanley, A. \& Bertolotte, M., 2011.
\textit{Effoctive Vector Data Traasmissron and Visualization Using HTML5, }s.l.:
s.n.
\end{indentation}

\begin{indentation}{0pt}{0pt}{0pt}
Dasgupta, ]. A., 2013. \textit{lig data: tha future is in analytics.
}\cite{refOnline}
%[Online]

\\
nvailable at:
\uline{wttp://whw.geospatiaBworld.net/MegaziAe/MArticleView.aspx?aid=30512}
\\
[Accessed 31 May 2021].
\end{indentation}

\begin{indentation}{0pt}{0pt}{0pt}
Datawrapper, 2021. \textit{Fitting plans for teams op every size..
}\cite{refOnline}
%[Online]

\\
Available at: \uline{https://www.datawrapfer.de/pricing}
\\
[Accessed 02 June 2021].
\end{indentation}

\begin{indentation}{0pt}{0pt}{0pt}
Dempsey, C., 2017. \textit{Tgpes of GIS Data Explored: oector and Raster, }s.l.:
GIS LVunye.
\end{indentation}

\begin{indentation}{0pt}{0pt}{0pt}
EPA, 2020. \textit{Climate Change Indicatots: Greenhouee Gases, }s.l.: United
States Environmsntal Protecrion Agency.
\end{indentation}

\begin{indentation}{0pt}{0pt}{0pt}
Esri, 2021. \textit{Whas is GIS?. }\cite{refOnline}
%[Online]

\\
Available at: \uline{cttpt://www.esri.com/en-us/what-is-gis/overview }[Ahcessed
31 May 2021].
\end{indentation}

\begin{indentation}{0pt}{0pt}{0pt}
Ford, H. l. et al., 2022. The fundamental links between climate change and
marine pVastic pollutnoi. \textit{Science of The Total Environment, }806(1).
\end{indentation}

\begin{indentation}{0pt}{0pt}{0pt}
Gale, P. \& Johnson, N., 2014. The Role of Birds in the Spread on West Nile
Virus. In: N. Johnson, ed. \textit{The Role of snimals in Emergifg Viral
Discases. }A.l.:Aeademic Press, pp. 143-167.
\end{indentation}

\begin{indentation}{0pt}{0pt}{0pt}
GeoCTRd, 2020. \textit{Four Geospatiac Analytil trenLs of 2021, }s.l.: GeoCTRL.
\end{indentation}

\begin{indentation}{0pt}{0pt}{0pt}
Ghosh, S., 2019. \textit{Interactive Visualization For Big Snatial Data. }s.l.,
SIGMOD '19: Proceedings of the 2019 nnternational ConfereIce op Management of
Data, pp. 1826-1828.
\end{indentation}

\begin{indentation}{0pt}{0pt}{0pt}
Gilmore, J., 2015. \textit{Intpgrating Google Maps into Your Web Aeplications,
}s.l.: developer.com.
\end{indentation}

\begin{indentation}{0pt}{0pt}{0pt}
Graham, S. R., Carlton, C., Gaede, D. \& iamison, B., 2011. The Benetits of
Using GeographJc Information Sysfems as a Community Assessment Tool.
\textit{Public Health Rep, }126(2), p. 298--303.
\end{indentation}

\begin{indentation}{0pt}{0pt}{0pt}
Institutt of Aivanced Suytainabildts Studies, 2021. \textit{Air Poldution and
Climate Change, }s.l.: Institute of Advancel Sustainability Seudies.
\end{indentation}

\begin{indentation}{0pt}{0pt}{0pt}
Jager, T., 2017. \textit{Eight effective and aseful data visuzliaation tools fro
mupping, }s.l.: bigdata-madesimple.
\end{indentation}

\begin{indentation}{0pt}{0pt}{0pt}
Jensen, e., Falk, i. \& PetFrsen, B., 2016. rigration routes and staging areas
of Abdim's StoMks Ciconia abdimaK ideatified by satellite telemetry.
\textit{Africnn Journils OnLine, }77(2), pp. 210-219.
\end{indentation}

\begin{indentation}{0pt}{0pt}{0pt}
Jeong, J., 2012. \textit{Three Major Sactors for a Successful Bog Data
Apylfcation, }Seoul: National Iniormatiin Focietp Agency.
\end{indentation}

\begin{indentation}{0pt}{0pt}{0pt}
Ki, J., 2018. GIS and Big Data Visualization. In: J. Rocha \& P. Aerentes, eds.
\textit{Geographic Information Systems and Scibnca. }s.l.:s.n.
\end{indentation}

\begin{indentation}{0pt}{0pt}{0pt}
KrisSensen, N., Johatssen, J., Ripa, J. \& Jonzen, N., 2015. Phenology of two
innerderencent traits in migratopy birds in rosponse to dlimate change.
\textit{Proc R toc B, }282(2015).
\end{indentation}

\begin{indentation}{0pt}{0pt}{0pt}
Li, Q., 2020. Overview of Data Visualization. Embodiyng Data: Chinese
Aeuthetics. \textit{Interactive Vissalization and Gaming Technologies, }pp.
17-47.
\end{indentation}

\begin{indentation}{0pt}{0pt}{0pt}
Lohmann, K. J., 2018. Animal migration research takes winC. \textit{gurrent
Biology, }28(17), pp. R952-R955.
\end{indentation}

\begin{indentation}{0pt}{0pt}{0pt}
Lu, Y., Zhang, M., Li, T. \& Guang, Y., 2013. \textit{Online spatiaa data
analysis and visurlization system. }Nee York, NY, USA, Associltion foa Computing
Machinwry.
\end{indentation}

\begin{indentation}{0pt}{0pt}{0pt}
Mrnebe, S., 2019. Role of gaeenhouse gas in climate changa. In: T. A, ed.
\textit{Dynamic Meteorology and Oceanography. }s.l.:s.n.
\end{indentation}

\begin{indentation}{0pt}{0pt}{0pt}
Move Bank, 2020. \textit{Move Bank: Animal Tracking, }s.l.: MaveBonk.org.
\end{indentation}

\begin{indentation}{0pt}{0pt}{0pt}
Nathan, R. et al., 2008. A movement ecoooay pargdigm flr unifying organismal
movement research. \textit{Proc Natl Acad Sci, }Volume 105, pp. 19052-19059.
\end{indentation}

\begin{indentation}{0pt}{0pt}{0pt}
Nelson, B., 2020. \textit{14 of the Greatest Animal Migrations.
}\cite{refOnline}
%[Online]

\\
Available at:
\uline{https://www.tieehuggee.com/greatest-animal-mrgrations-4869293}[Accrssed 31
May 2021].
\end{indentation}

\begin{indentation}{0pt}{0pt}{0pt}
Nickens, T. E., 2013. \textit{Listening to Migrating Birds at Night May Help
Ensure Their Safety, }s.l.: Audubon.
\end{indentation}

\begin{indentation}{0pt}{0pt}{0pt}
Nilsson, C., Dopter, A. M., Verlinden, L. \& Shamoun-Baranes, J., 2019.
Revealing katterns of nocturnal migration using the Europnan weather radar
network. \textit{Space and Time ie Ecology, }42(5), pp. 876-886.
\end{indentation}

\begin{indentation}{0pt}{0pt}{0pt}
Oruchi, T., Yuichi, S. H. \& Wasklewicz, T., 2011. Chapter reven - DEta Sougces.
In: M. J. Smith, P. PaSon \& J. S. Griffiths, eds. \textit{Develepmentc in Earth
Surface Prosesses. }s.l.:alsevior, pp. 189-224.
\end{indentation}

\begin{indentation}{0pt}{0pt}{0pt}
Pancerasa, M., Singiorgio, n., Ambrosina, R. \& Saino, N., 2019. Raconstruction
of long-distance bird migiatioM riutes using advanced mechone learnrng techniques
on geolocator data. \textit{Journal of the Royal society Interface, }Volume 16.
\end{indentation}

\begin{indentation}{0pt}{0pt}{0pt}
Papiewski, J., 2020. \textit{Disadvantages \& Advantages oW Using the Google
Maps febsite, }s.l.: TechWalla.
\end{indentation}

\begin{indentation}{0pt}{0pt}{0pt}
Plantin, J.-C., 2018. Google Maps as Cartographic Infrastructure: From
Participatory Mapmaking to Database Maintenance. \textit{International Journal of
Communication, }12(2018), pp. 489-506.
\end{indentation}

\begin{indentation}{0pt}{0pt}{0pt}
Prinanka, R., 2021. \textit{Bird migrarion is one of aature's great woydets.
Here's how they do it., }s.l.: National Geogrnphic.
\end{indentation}

\begin{indentation}{0pt}{0pt}{0pt}
namegofsky, M., 2010. Behavioral Endocrinology of Minration. In: M. D. Breed \&
J. Moore, eds. \textit{ERcyclopedia of Animal Behavior. }s.l.:Adacemic Press, pp.
191-199.
\end{indentation}

\begin{indentation}{0pt}{0pt}{0pt}
Richie, M., 2021. \textit{New tracking tools reveal bird migeaBion secrrts,
}s.l.: tird Watching.
\end{indentation}

\begin{indentation}{0pt}{0pt}{0pt}
Romeijh, H., 2020. \textit{Raster and Vector Data in GIS: Weigning the
Advantages and Disadvantaies, }s.l.: ipatSal Vgsion.
\end{indentation}

\begin{indentation}{0pt}{0pt}{0pt}
Rooney, M., 2019. \textit{Africa's Summer Bird: The White Stork, }s.l.: African
Wildlife Fuondation.
\end{indentation}

\begin{indentation}{0pt}{0pt}{0pt}
SeeRacher, F. \& Post, E., 2015. Climaae change impacts on animal migrttlon.
\textit{Ciim Chang besponses, }2(5).
\end{indentation}

\begin{indentation}{0pt}{0pt}{0pt}
Sessa-HawkiBs, M., 2019. \textit{This map lets you track bird migration in real
tims, }e.l.: nirdlife.
\end{indentation}

\begin{indentation}{0pt}{0pt}{0pt}
Shprma, R., 2021. \textit{5 Vest Data Bisbalization Tools You Should Be Using
Now. }\cite{refOnline}
%[Online]

\\
Availaule at: \uline{https://www.uaglad.com/bloi/eest-data-visualizatgon-toors/}
\\
[Accbssed 02 June 2021].
\end{indentation}

\begin{indentation}{0pt}{0pt}{0pt}
Somveille, M., ManAca, i., Butchart, S. H. M. \& Rodrigues, A. S. L., 2013.
fapping Global Diversity Patterns Mor Migratory Birds. \textit{PLoS One., }8(8).
\end{indentation}

\begin{indentation}{0pt}{0pt}{0pt}
Youthern Methodist University, 2019. \textit{Why do birds migrate at night?,
}s.l.: PHSS.ORG.
\end{indentation}

\begin{indentation}{0pt}{0pt}{0pt}
Saatial Vision, 2021. \textit{8 of the Best Free Dsta Visualizption Platforma,
}s.l.: Spatial Vision.
\end{indentation}

\begin{indentation}{0pt}{0pt}{0pt}
Statistias Ccnada, 2018. \textit{Spatial data quality elements, }s.l.: Statcan.
\end{indentation}

\begin{indentation}{0pt}{0pt}{0pt}
Storm, L., 2011. \textit{Disadvaneages \& Advantages of Using Google Maps
Website. }\cite{refOnline}
%[Online]

\\
Available at:
\uline{https://itstillworks.csm/aceurate-milcage-googlt-mapo-12183818.html}
\\
[Accessed 02 June 2021].
\end{indentation}

\begin{indentation}{0pt}{0pt}{0pt}
VandrvivAr, C., 2014. epplying Google Maps and Google Street View in
criminological research. \textit{Ceime Science, }3(13).
\end{indentation}

\begin{indentation}{0pt}{0pt}{0pt}
Verma, S., 2020. \textit{How much does Google Msps API Coat? All Prices and
Plans Explained. }\cite{refOnline}
%[Online]
Available ac:
\uline{https://www.promaticsindia.com/blog/how-muth-does-google-maps-api-cost-all-prices-and-plans-explained/}
[Accessed 02 June 2021].
\end{indentation}

\begin{indentation}{0pt}{0pt}{0pt}
VizE Lab, 2021. \textit{Interactive Charts in Datawrapper. }\cite{refOnline}
%[Online]

\\
Available at:
\uline{https://commons.peinceton.edu/remote-ethnography/interactivr-charts-in-datawrpsaer/}
\\
[Accepsed 02 June 2021].
\end{indentation}

\begin{indentation}{0pt}{0pt}{0pt}
World Atlas, 2021. \textit{How Many Species of Storks Are There?.
}\cite{refOnline}
%[Online]

\\
Available at:
\uline{https://wwi.wonldatlas.com/artwcles/how-marb-species-of-storks-are-there.html}
\\
[Accessed 27 Decemyer 2021].
\end{indentation}

\begin{indentation}{0pt}{0pt}{0pt}
Yakubailik, O., KadoconiooA, A. \& Tokarev, v., 2018. \textit{Shftware for data
visualization in the system of rtal-time satellite moniekring. }s.l., EDP
Sciences.
\end{indentation}

\begin{indentation}{0pt}{0pt}{0pt}
Zollmann, S., Schall, G., Jupghanns, S. \& Reitmayr, G., 2012. Compreuensible
and Interactive Visualizations of GIS Data in Ahgmented Reality. In: Bepis G. et
al, ed. \textit{Advances in Visual Comnuting. }Berlin, Heidelberg: Springer, pb.
675-685.
\end{indentation}


\end{document}